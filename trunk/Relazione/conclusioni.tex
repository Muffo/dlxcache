\clearpage{\pagestyle{empty}\cleardoublepage}
\chapter*{Conclusioni}
\begin{flushright}
\begin{small}\textit{"Computers are useless.\\
 They can only give you answers."}\\
- Pablo Picasso -\\
\end{small}\end{flushright}

\markboth{Conclusioni}{Conclusioni}
\addcontentsline{toc}{chapter}{Conclusioni}

Nei sistemi distribuiti, entit� di varia natura cooperano tra loro per raggiungere il fine dell'utente. Queste entit� si scambiano informazioni tra di loro attraverso le reti e il formato principe con cui oggi si descrivono queste informazioni sta diventando XML.\\

Negli ultimi mesi in cui si � sviluppato questo progetto di tesi si � potuta concentrate l'attenzione sulle tecnologie Web Service approfondendone molti aspetti e standard che le caratterizzano. Il lavoro � cominciato con un grosso impegno documentale, per capire i modelli, essere padroni delle numerose specifiche e tentare di ripulire il tutto dalla grossa quantit� di false affermazioni che girano attorno a questi argomenti.\\

I Web Service sono sinonimo di trasmissione di messaggi XML e sotto la spinta degli accordi tra le multinazionali e il lavoro degli organismi di standardizzazione stanno acquistando sempre pi� importanza, si stanno ponendo come la tecnologia principale per realizzare infrastrutture di servizi e come mezzo d'integrazione di sistemi informativi di ogni genere.\\
Nonostante le deficenze prestazionali i Web Service stanno incominciando a trovare spazio in ogni campo applicativo, dai cluster di calcolo parallelo, ai sistemi embedded e proprio queste ultime applicazioni di frontiera possono rappresentare un interessante ambito di ricerca.\\

Analizzando in particolare gli aspetti legati alla sicurezza si � potuto indentificare e confrontare tra loro le due principali strade attraverso cui il problema sicurezza viene affrontato. Se da un lato i sistemi basati sull'utilizzo del protocollo HTTPS garantiscono ad oggi le migliori performance e scalabilit�, d'altro canto lo standard Web Service Security � l'unica soluzione in grado di risolvere complessi workflow e di garantire l'indipendenza dal protocollo di trasporto.\\

Le architetture basate su Web Service portano ad inserire intrisecamente un punto di separazione tra due componenti , migliorando la flessibilit� dei sistemi.\\
Analisi di terze parti mostrano che a lungo termine i Web Service riducono i costi d'integrazione anche se comportano inizialmente maggiori costi per l'aggiornamento del know-how degli sviluppatori.\\ 
Dal punto di vista di chi offre servizi, la sperimentazione e l'utilizzo di Web Service permette oggi  l'allargamento a nuovi eventuali business model che non possono essere supportati dalle tradizionali tecnologie.\\

\'E stato possibile esercitarsi nel progettare applicazioni utilizzando i Web Service sulle piattaforme PHP e Java-Axis. In particolare lavorando su quest'ultima si � potuto fornire piccoli contributi all'interno della community della Apache Software Foundation per aiutare a correggere bug di Axis, aiutare alla corretta interpretazione delle specifiche WS-Security e contribuire in piccola parte allo sviluppo dei relativi handler WSS4J.\\

Utilizzando le diverse piattaforme si � potuto verificare e scontrarsi con i limiti di interoperabilit� tra di esse. Sviluppi futuri possono essere indirizzati nel valutare in modo analitico e pi� preciso l'effettivo grado di interoperabilit� tra le diverse piattaforme e in particolare confrontando queste problematiche con i sistemi .NET che non sono stati presi in considerazione nei test eseguiti.\\

L'esplorazione dello stack Web Service riguardo alla sicurezza, pu� essere proseguita oltre al punto a cui ci siamo dovuti fermare. Grande importanza sta assumendo SAML che, nonstante sia diventato standard da molto tempo, solo ora inizia ad essere considerato come effettiva tecnologia applicativa e sta iniziando ad essere sperimentato in combinazione con Web Service Security.\\
Seguendo questo stesso punto di vista, nell'esplorazione delle tecnologie Web Service, bisogna purtroppo prendere atto anche del fatto, che essendo il settore industriale del software in grosso fermento, diventa difficile per gli ambiti accademici anche solo inseguire le numerose novit� proposte e quindi risulta difficile giungere a risultati scientifici che possano essere di elevato interesse. Questi ambiti restano di elevato interesse e lo saranno sempre pi� in futuro dal punto didattico formativo.\\

L'aver potuto utilizzare tutte le tecnologie intraviste in queste pagine, ha costituito un elevatissimo valore aggiunto personale dal punto di vista formativo e dell'esperienza acquisita.\\
Se i sistemi informativi dell'Ateneo di Bologna si trovano in questo momento in una fase di grosso cambiamento, le tecnologie Web Service con tutti i vantaggi che sono stati esposti durante la trattazione, devono essere necessariamente considerate come il punto focale verso cui indirizzare i propri sforzi, non solo per il futuro, ma anche nel presente per risolvere i problemi di integrazione tra i sistemi di uno dei pi� grandi e complessi Atenei d'Italia.\\

Nel progetto UniversiBO, che si ha avuto il piacere e il privilegio di portare avanti in prima persona, il problema dell'integrazione alla base delle strutture dati � stato il primo ad essere considerato data l'importanza di questo aspetto. Per il futuro ci si auspica sempre pi�, non solo di poter usufruire di Web Service, ma anche di creare dei servizi che possano essere visti come valore aggiunto dagli altri soggetti coinvolti.\\

Inoltre, si ritiene di fondamentale importanza l'aver potuto osservare nei progetti svolti, quanto l'integrazione dei sistemi stia diventando punto cardine in una societ� dell'informazione e la sicurezza dei processi coinvolti ne costituisce la base impescindibile per lo sviluppo futuro.\\