\clearpage{\pagestyle{empty}\cleardoublepage}
\chapter*{Conclusioni}

\markboth{Conclusioni}{Conclusioni}
\addcontentsline{toc}{chapter}{Conclusioni}

Durante l'attivit\`a progettuale abbiamo realizzato una cache per il processore DLX.\\
In particolare si \`e realizzata una cache di tipo set-associative con numero di vie e dimensione configurabili. Il componente \`e stato realizzato in VHDL ed \`e testato individualmente sfruttando l'ambiente integrato di Xilinx.\\

La cache \`e stata poi integrata all'interno del progetto del DLX. Per verificare il corretto funzionamento della nuova versione del processore si sono scritti diversi programmi in assembler che accedono ai dati presenti nella cache realizzata.	\\

Infine si \`e analizzato il funzionamento della Block RAM presente all'interno dell'FPGA, grazie alla quale \`e possibile realizzare una RAM di dimensioni molto superiori.

