\clearpage{\pagestyle{empty}\cleardoublepage}
\chapter{UniversiBO v2: alcune specifiche} 
\label{appendiceRequisiti}  
In questa appendice riportiamo alcuni esempi di casi d'uso, di user stories e alcuni feedback ricevuti dagli utenti.\\

\subsubsection{Casi d'uso}
Questi a seguire, sono alcuni casi d'uso UML costruiti all'inizio della progettazione.\\
Descrivono alcune propriet� generali d'uso generali che stanno alla base della logica applicativa.\\

L'utente entra nell'applicazione per ricercare informazioni, queste sono raggruppate su canali tematici, per cui navigando raggiunge il canale di suo interesse e ne fruisce dei servizi.\\

\begin{figure}[!ht]
 \centering
\includegraphics{img/40-usecase-autenticazione.png}
 \caption{Use case autenticazione}
 \label{fig:usecase-autenticazione}
\end{figure}

L'utente deve potersi autenticare per poi poter usufruire dei servizi personalizzati.\\

\begin{figure}[!ht]
 \centering
\includegraphics{img/41-usecase-info-servizi.png}
 \caption{Use case ricerca e fruizione servizi}
 \label{fig:usecase-info-servizi}
\end{figure}

Dopo l'autenticazione che permettere di identificare al sistema i diversi attori che possono svolgere azioni diverse sui servizi, disponibili.\\
L'autenticazione permette di personalizzare anche la navigazione per una ricerca pi� veloce dei canali di interesse.\\

\begin{figure}[!ht]
 \centering
\includegraphics{img/42-usecase-servizi.png}
 \caption{Use case ricerca e fruizione servizi}
 \label{fig:usecase-servizi}
\end{figure}


\subsubsection{Diagrammi di navigazione}
Lo studio dei processi di navigazione del sito parte con un diagramma di navigazione principale.\\
Questo schema a blocchi descrive le principali modalit� con cui l'utente deve poter raggiungere direttamente l'informazione desiderata e costituisce lo scheletro di base su costruire la navigazione all'interno del sito. Purtroppo non esistono strumenti standard per descrivere questo processo.\\

\begin{figure}[!ht]
 \centering
\includegraphics{img/43-digramma-navigazione.jpg}
 \caption{Diagramma di navigazione}
 \label{fig:diagramma-navigazione}
\end{figure}

Nella figura \ref{fig:diagramma-navigazione} il blocco MyUniversiBO rappresenta una vista personalizzata sui blocchi informativi principali.\\

A partire da questo diagramma si costruiscono le prime bozze grafiche, eventualmente si formalizza ulteriormente il processo di navigazione di alcune sottoparti e infine si fanno delle analisi di usabilit� delle singole parti.\\
Il processo andrebbe poi ulteriormente raffinato facendo analisi statistiche sul sito in produzione e raccogliendo i problemi e difficolt� pi� comuni in cui incorrono gli utenti.\\

\subsubsection{User stories}
Per esempio per alcuni servizi come i Files o le News i requisiti sono stati inizialmente raccolti in base alla precedente versione, feedback e il benchmarking.\\

\begin{quotation}Contenuti: Titolo, Notizia, Data di inserimento, Autore\\
News assegnabili a canali ed aree di interesse, una notizia pu� appartenere a pi� canali\\
Possibilit� di visualizzare gruppi di dimensioni diverse delle ultime n-notizie\\
Possibilit� di scrivere, cancellare, modificare le news\\
Possibilit� di notificare le notizie, con diverse priorit� (urgente/non urgente)\\
Politiche dei diritti per scrittura modifica, cancellazione:\\
- Moderatore: Modifica le news di cui � proprietario negli argomenti che modera,\\ Scrivere negli argomenti che modera\\
- Referente: Pu� scrivere e modificare tutto nei sui argomenti\\
- Admin: Gli � permesso tutto in tutti i canali\\
- I referenti e gli admin possono delegare i diritti ad altri\\
Possibilit� di segnalare le notizie pi� nuove rispetto ad una certa data\\
Possibilit� di inserire news con data posticipata\\
Possibilit� di inserire news con scadenza (con visualizzazione opzionale per chi ha il\\ diritto di modificarle/cancellarle)
Possibilit� di utilizzare codici speciali (bbcode) per inserire oggetti grafici (come faccette, ecc?)\\
Possibilit� di riconoscere automaticamente i link all'interno del testo della news e visualizzarli come tali\\

Feedback: i docenti si sono lamentati di non poter inserire contemporaneamente in pi� canali una notizia.\\
Feedback: � noioso quando accade un errore nell'inserimento dover reinserire da capo tutti i dati del form.\\
Feedback: nelle pagine di inserimento e modifica dovrebbe comparire un help a portata di mano.\\
Feedback: quando inserisco una notizia a volte i tempi d'attesa sono elevati (il problema � dovuto al fatto che le notifiche e-mail vengono inviate in maniera sincrona, bisogna renderlo asincrono).\\
\end{quotation}

Dopo una prima analisi sommaria del design pi� adatto al componente trattato, si sono scritte delle user stories che ricoprissero tutti i requisiti.\\
Ogni user story corrisponde ad un task legato ad una funzionalit� concreta dell'applicazione. Come prevedono le metodologie agili, lo scopo � quello di scegliere sempre la funzionalit� con il maggiore valore aggiunto per il prodotto e che abbia un riscontro effettivo per l'utente finale.\\
Ogni user story � stata appuntata su un cartoncino di piccole dimensioni � presa in carico da un membro del team che si occuper� di implementarla.\\
Un esempio significativo di user story � il seguente:\\

\textit{Data: 23-01-2004\newline
\newline
Visualizzando un canale devono essere mostrate le ultime N notizie appartenenti ad esso.\newline
Se una delle notizie � nuova rispetto all'ultimo accesso dell'utente deve comparire un'immagine grafica per distinguerla.\newline
Se l'utente possiede i diritti necessari deve vedere il link per aggiungere una notizia e accanto ad ogni notizia i link per modificarla e/o eliminarla.\newline
Se ci sono pi� di N notizie deve comparire un link ad un archivio della pagina.\newline
\newline
Implementata da: brain\newline
Data: 03-03-2004\newline
\newline
Note:\newline
Ho aggiungiunto alla classe NewsItem il campo \$username e i relativi metodi acessori, duplicando la logica di User. Non � carino ma � molto pi� veloce.\newline
Se il canale � tra i preferiti prendo la data dell'ultimo accesso al canale, se non � tra i preferiti \_non\_ prendo l'ultimo login.\newline
Bisogna implementare le operazioni di modifica/elimina e mostrare l'archivio.\newline
Bisogna scrivere il contenuto dell'help.}\\ 

I cartoncini utilizzati sono scelti appositamente di piccole dimensioni per costringere a creare piccoli task.\\
Uno sviluppatore prendendosi l'incarico di svilupppare una carta ne assume anche il possesso fisico che serve a stimolarlo maggiormente e serve da promemoria.\\


