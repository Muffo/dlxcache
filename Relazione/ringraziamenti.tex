\clearpage{\pagestyle{empty}\cleardoublepage}
\chapter*{Rigraziamenti}
\fancyhf{} %Clears all header and footer fields, in preparation.
\addtolength{\parskip}{- 5pt}
In poche righe � difficile ricordare tutti coloro che in questi anni mi sono stati vicini.\\
Un'abbraccio speciale va ad Andrea, Marco e Matteo per aver condiviso con me la maggior parte delle sfide attraverso cui siamo passati, per avermi permesso di crescere al vostro fianco e soprattutto per tutto l'affetto che mi avete sempre dimostrato. Un'abbraccio altrettanto speciale va ad Alessandro per aver rappresentato per tanti anni la valvola di sfogo del mio furore dionisiaco e nonstante tutto, anche il punto di riferimento su cui poter sempre contare. Questo traguardo � anche il vostro.\\

Desidero ringraziare il prof. Antonio Corradi per avermi aiutato nell'affrontare ogni situazione critica di questo progetto con la massima tranquillit�, per la disponibilit� sempre dimostrata e per essere stato il riferimento culturale che ogni studente vorrebbe avere.\\
Ringrazio il prof. Andrea Zanoni per tutto il supporto, la fiducia e la stima che mi ha dimostrato in questi anni.\\
L'ing. Enrico Lodolo per tutti i preziosi consigli e perch� molte delle idee nate durante questa tesi sono nate, in un modo o nell'altro, dalla possibilit� di potermi confrontare con una persona dalle sue ammirevoli capacit�.\\
L'ing. Luca Ghedini per tutti gli affettuosi consigli e per avermi dato la possibilit� di conoscere una persona cos� speciale.\\

Grazie alla mia famiglia per tutto l'affetto e il sostegno. In particolare un ringraziamento al contrario per mia mamma, perch� con tutto lo stress a cui mi ha sottoposto ormai sono diventato invulnerabile, a mio padre per essere riuscito a riequilibrare la bilancia e a mio fratello per aver rappresentato la mia guida e il mio modello di riferimento.\\
Un grandissimo grazie ai miei nonni, che anche se ormai scomparsi, sono stati i miei primi supporter e tifosi del mio buon andamento scolastico. Mi piace pensare a quanto sareste stati fieri di me se foste ancora qui. Grazie ai vostri insegnamenti sono arrivato fino a questo traguardo\ldots  e non intendo certo fermarmi!\\
Grazie a tutti i miei parenti, agli zii vicini e lontani e ai miei cuginetti: Gerardo, Nina, Rosalba, Michele, Marco e Maurizio!\\

Grazie a tutti coloro che mi hanno aiutato a correggere e revisionare queste pagine ed in particolare: mio fratello, Ivana, Fabrizio, Matteo, Davide, Elisa, Marco, Andrea. Il vostro sostegno mi rende ancora pi� fiero di esservi amico.\\

Grazie a McFabbri, perch� nonstante sia stato per 3 anni mio compagno di stanza, non � riuscito ad attaccarmi l'influenza!\\
Grazie a Marta e Lele per avermi permesso di incontrare SuperBlindStatoMarcoWar! Grazie a Vampeta, a quell'anno sabbatico, a Bob, alle serate Blind, alla banda e all'ironia!\\
Grazie tante a Nicola, anche se avremmo avuto piacere di vederti un po' pi� spesso in casa tua!\\

Grazie ai cari amici che hanno allietato questi miei cinque (beh... sei) anni passati su libri e appunti: Pisi, Il Ghedo, Davidone, Gabba, Stefano, Denis, l'ing. Lucchi, Max, Yu, Luca, Bianconi, Roberto, Ivo, Marco Cova, Marco Corvini, la mitica Manu, Sara, la Cappi, Roberta, Enrica, Nadia e il Cavvaaa!!\\

Grazie a tutti gli altri ragazzi con cui ho avuto il piacere di condividere l'esperienza di UniversiBO, non posso ricordarvi tutti e 101, per cui oltre a quelli gi� citati ci tengo almeno a ricordare: Elena, Micol, Silvia, Anna Chiara, Daniele (Rocco), Daniele (LastHope), Luigi (Buddolo), Francesco (fufu), Luciano, Domenico, Vincenzo (Mel), Giuseppe, Roberto, Andrea, Ermete e Nicola. Da domani tutto � nelle vostre mani: conto su di voi!\\

Grazie a tutti i miei vecchi compagni di classe, perch� nonostante la distanza che ci ha divisi, vi sento ancora tutti vicini ed uniti, per cui dai tempi del liceo vi ricordo ancora tutti in ordine alfabetico come una formazione di calcio: Albo, Baccarini, Bartolini (io), Benini, Berdondini, Bertozzi, Cani, Cani, Ceroni, Clo', Ferrini, Folli, Gattelli, Giunchedi, Grandi, Lugatti, Pagnini, Tugnoli\ldots in panchina Nicola!\\

Grazie ai colleghi di lavoro del CILTA che in questo ultimo anno, hanno sopportato le mie fugaci apparizioni tra una parte e l'altra di stesura di questa tesi. In particolare grazie agli occhi pi� belli del CILTA: Orsola! Grazie a tutti ragazzi dello STIC, a Roberto, Riccardo, Fabio, la dolce Valentina e alle meravigliose ragazze dell'Amministrazione. Grazie ai colleghi della piccola parentesi milanese Nicola, Lucia e Simone.\\

Un grazie speciale a Greta per avermi mostrato un nuovo lato dell'amicizia. Grazie alla Manu, Francy, Lucy, Gloria e Federica. Grazie a Bighli, Mirko e Ortes. Grazie a Rece, Ciupa e Pippo. Grazie all'imperatrice Lety, al mitico Mattia e a Simone. Grazie a Claudio e Luigi. Grazie a Poggi, Valeria e Gigiaz. Grazie a Stefano e Gigio. Grazie a Sergio, Bruno, Vass, Poggi (Luca il cugino), Mich e Bruce.\\

Grazie a tutti coloro che in passato sono stati miei insegnanti e mi hanno permesso di imparare nuove lezioni, dalle equazioni differenziali a come raccogliere le pere nei campi, nella scuola e nella vita. Grazie senza rimpianto ai miei passati amori, per tutto ci� che abbiamo vissuto e perch� proprio per questo, domani, mi potr� svegliare con la sensazione di sentirmi libero!\\

Mi ero imposto di far restare questi ringraziamenti in una pagina, ma non ci sono riuscito, quindi una nota speciale va a tutti coloro che non si offenderanno per non essere presenti neanche nella seconda e nella terza\ldots ma il mio affetto in questi anni � cresciuto cos� tanto che non mi sono affatto dimenticato di tutti voi. Ricordarvi stasera mentre scrivevo queste pagine che chiudono un altro capitolo della mia vita � stata per me un'immensa gioia!\\
Anche per tutti voi uno speciale: ``grazie grazie grazie!!''.\\



% A parte vorrei lasciare alcune conclusioni sul progetto UniversiBO, cha anche se costituisce solo l'incipit di questa tesi, ho avuto il privilegio di portare avanti dai primi mesi del 2002 in tutte le sue fasi ed in tutti i suoi aspetti.\\
% 
% A partire dal concept, la raccolta dei requisiti, la progettazione e l'implementazione della prima versione del sito ho avuto la possibilit� di lavorare per la prima volta in team.\\
% Essendo il mio primo progetto personale, nato da poca esperienza, ho potuto subito provare sulla mia pelle il processo di degenerazione delle architetture software.\\
% 
% Dal giugno 2003 � partita la reingenerizzazione di tutto il servizio, con la sfida di ricominciare tutto da capo.\\
% La piattaforma � riuscita ad integrare numerosi componenti open source, a creare al suo interno componenti d'avanguardia riutilizzati gi� in altri progetti. \'E riuscito a mettersi in luce a livello nazionale come uno dei migliori progetti realizzati con le strategie del software libero nella sua categoria.\\ 
% 
% L'architettura � in grado di ospitare i contenuti informativi di uno dei pi� grandi Atenei d'Italia come quello di Bologna ed � in grado di essere facilmente integrata con la didattica di qualsiasi altro Ateneo tramite un'interfaccia semplice e completa.\\
% La parte web della piattaforma � disegnata secondo avanzati criteri di usabilit�, con tecnologie d'avanguardia come CSS2 e con il pieno rispetto delle normative per l'accessibilit� dei diversamente abili.\\
% Oggi UniversiBO � un ottima piattaforma per il blended-learning, che ospita una community basata sulla condivisione del sapere. Oggi conta oltre 4000 iscritti, 3900 comunicazioni e 2400 documenti pubblicati on-line, 19000 messaggi scambiati tra studenti ed oltre 2500 visite giornaliere.\\
% 
% In questo progetto ho potuto sudare affrontando il lavoro "sporco". Mi sono trovato un lavoro part-time per guadagnare i soldi per acquistare un server attraverso cui poter fornire il servizio. Ho passato notti insonni ad installare e configurare il server, ricopiare dati e tabelle, tracciare diagrammi UML, scrivere oltre 5MB di codice sorgente, 8MB di documentazione e tanto altro che non val la pena quantificare in byte.\\ 
% 
% Ho avuto il privilegio di affrontare con passione ed un pizzico d'ironia queste sfide, condividendole con ottimi amici che mi hanno insegnato come migliorare me stesso e risolvere molti aspetti del progetto.\\
% Mi sono impegnato ad organizzare un gruppo di persone che ha raggiunto le oltre 50 unit�, a studiare ed applicare metodologie per coordinarle in modo compatibile con l'ambiente sia off-line che on-line.\\
% Ho potuto affrontare problematiche legali, burocratiche e politiche cha hanno coinvolto un progetto di questa portata.\\
% Ho potuto apprezzare l'importanza degli aspetti di comunicazione con l'esterno e all'interno del team. Imparare ad ascoltare e ad esprimere le cose giuste.\\
% Mi sono impegnato a fondo per trasmettere le mie conoscenze implicite ed esplicite, in forma orale e scritta, formale ed informale ad altri membri del progetto. Ho imparato l'importanza della conoscenza in un'ambiente competitivo ed in forte evoluzione.\\
% 
% In questi tre anni di progetto ho avuto la possibilit� di approfondire ed imparare tecnologie e strumenti che non facevano parte del mio curriculum accademico.\\
% Ho imparato l'importanza di mettersi in gioco ed impegnarsi in prima persona, come mezzo per guadagnare la stima reciproca dei propri colleghi. Ho imparato a condividere con loro difficolt� e momenti di tranquillit�.\\
% 
% Riprendendo l'aforisma di Picasso all'inizio di questo capitolo, soprattutto ho imparato ad analizzare ed affrontare i problemi cercando le domande giuste prima della soluzione.\\
%  
% (CONTINUARE).\\
