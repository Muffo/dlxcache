\clearpage{\pagestyle{empty}\cleardoublepage}

\chapter{Testbench}

Per morlins

\section{Testbench del componente}

I tipi di dati utilizzati sono definiti nel file \texttt{Cache\_lib.vhd}.

\lstset{language=VHDL, caption=Costanti e tipi di dato definiti nel file \texttt{Cache\_lib.vhd}, label=DescriptiveLabel, breaklines=true, basicstyle=\small, showspaces=false, showtabs=false, stringstyle=\ttfamily, showstringspaces=false,  tabsize=3} % basicstyle=\tiny\ttfamily}

\begin{lstlisting}

codice...

\end{lstlisting}


\section{Assembler per DLX}

Dopo avere testato il funzionamento della cache e della ram singolarmente, si \`e passati al test del corretto funzionamento della cache inserita all'interno del progetto del processore DLX.
Per far ci\`o sono stati realizzati una serie di programmi in assembler, di cui  mostreremo solo i due significativi:
\begin{itemize}
 \item \texttt{provaReplacement123}:nel quale si verifica la corretta comunicazione tra cache e DLX e il meccanismo di rimpiazzamento.
 \item \texttt{provaFU}:nel quale si verifica il corretto funzionamento della Forwarding unit. 
\end{itemize}
\subsection{dal codice all' escuzione}
Per completezza in questa sezione si spiegher\`a brevemente come poter mettere in esecuzione un codice.
Per prima cosa si scrive il codice in assembler all'interno di un file con estensione *.dls, che viene poi dato in pasto all' assemblatore DASM, il quale lo converte in codice macchina mediante il comando(dal prompt di comandi windows):
\lstset{language=VHDL, caption=Costanti e tipi di dato definiti nel file \texttt{Cache\_lib.vhd}, label=DescriptiveLabel, breaklines=true, basicstyle=\small, showspaces=false, showtabs=false, stringstyle=\ttfamily, showstringspaces=false,  tabsize=3} % basicstyle=\tiny\ttfamily}

\begin{lstlisting}

dasm -a -l <nome_file>.dls

\end{lstlisting} 

il risultato sar� un file \texttt{<nome\_file>.dlx} che a sua volta dovr\`a essere convertito mediante la classe java \texttt{DLXConv}, per avere un file  \texttt{<nome\_file>.dlx.txt} contenente il codice in un formato direttamente inseribile all'interno del progetto del DLX.

In particolare dovr� essere inserito nel file \texttt{Fetch\_Stage.vhd} all'interno dell'array che sostituisce la EPROM contenente le istruzioni in linguaggio macchina, da dare in pasto al processore:
\lstset{language=VHDL, caption=Costanti e tipi di dato definiti nel file \texttt{Cache\_lib.vhd}, label=DescriptiveLabel, breaklines=true, basicstyle=\small, showspaces=false, showtabs=false, stringstyle=\ttfamily, showstringspaces=false,  tabsize=3} % basicstyle=\tiny\ttfamily}

\begin{lstlisting}

constant EPROM_inst: eprom_type(0 to 11) := ( 
-- istruzioni in linguaggio macchina.
);

\end{lstlisting} 

Ora si analizza i due codici pi\`u significativi nel dettaglio.
Per comodit\`a si riporter\`a il codice contenuto nella \texttt{EPROM_inst}  corredato di commento e codice assembler relativo.

\subsection{provaReplacement123}
\lstset{language=VHDL, caption=Costanti e tipi di dato definiti nel file \texttt{Cache\_lib.vhd}, label=DescriptiveLabel, breaklines=true, basicstyle=\small, showspaces=false, showtabs=false, stringstyle=\ttfamily, showstringspaces=false,  tabsize=3} % basicstyle=\tiny\ttfamily}

\begin{lstlisting}
X"20010000",	--l1: addi r1,r0,0 ; azzera r1
X"20020001",	--l2: addi r2,r0,1 ; imposta a 1 r2
X"AC220000",	--l3: sw 0(r1),r2 ; memorizzza il valore di r2 all'indirizzo 0+r1(via 1 dell index0)
X"20420001",	--l4: addi r2,r2,1 ; incrementa r2
X"AC220100",	--l5: sw 16#100(r1),r2 ; memorizzza il valore di r2 all'indirizzo 16#100+r1(via 0 dell index0)
X"20420001",	--l6: addi r2,r2,1 ; incrementa r2
X"AC220080",	--l7: sw 16#80(r1),r2 ; memorizzza il valore di r2 all'indirizzo 16#80+r1(replacement via 1 dell index0) 
X"8C220000",	--l8: lw r2,0(r1) ; ripristina valore iniziale di r2 (1)
X"20210004",	--l9: addi r1,r1,4 ; incremento di 4 indirizzo di base in r1
X"0BFFFFE0",	--l10: j l3 ;
X"FFFFFFFF",	--NOP 
X"FFFFFFFF" 	--NOP

\end{lstlisting} 
\subsection{provaFU}
\lstset{language=VHDL, caption=Costanti e tipi di dato definiti nel file \texttt{Cache\_lib.vhd}, label=DescriptiveLabel, breaklines=true, basicstyle=\small, showspaces=false, showtabs=false, stringstyle=\ttfamily, showstringspaces=false,  tabsize=3} % basicstyle=\tiny\ttfamily}

\begin{lstlisting}
X"AC22000A",  --l1: sw 10(r1),r2  ; salva il contenuto di r2
X"8C23000A",  --l2: lw r3,10(r1)  ; porta in r3 il valore presente in r2
X"20620001",  --l3: addi r2,r3,1  ; incrementa r2
X"0BFFFFF0",  --l4: j l1          ; salta a l1
X"FFFFFFFF",
X"FFFFFFFF",

\end{lstlisting} 

%Il numero di bit di offset, indice e tag \`e stato parametrizzato per rendere pi\`u flessibile l'utilizzo del componente.

%All'interno di \texttt{Cache\_lib.vhd} sono poi stati definiti i seguenti tipi di dati:
%\begin{itemize}
%  \item \texttt{data\_line}: contiene i dati per una linea della cache, la cui dimensione \`e calcolata in base al numero di bit di offset;
%  \item \texttt{cache\_line}: record contenente le informazioni su dati e stato di una linea;
%  \item \texttt{set\_ways}: array di \texttt{NWAY} linee che compongono una via;
%  \item \texttt{cache\_type}: array di vie, costituisce l'intera cache ??? (non so come scrivere... :S).
%\end{itemize}

%Per ogni \texttt{cache\_line} si tiene quindi traccia di:
%\begin{itemize}
%  \item \texttt{data}: \texttt{data\_line} relativa alla linea corrente;
%  \item \texttt{status}: indica lo stato MESI della linea;
%  \item \texttt{tag}: bit dell'indirizzo che rappresentano il tag della linea;
%  \item \texttt{lru\_counter}: contatore usato dalla politica di rimpiazzamento.
%\end{itemize}

%
%\begin{figure}[h!]
%\centering
%\includegraphics[width=\textwidth]{img/cacheType.png}
%\caption{Schematizzazione delle strutture dati della cache}
%\label{fig:c_type}
%\end{figure}

%In Fig. \ref{fig:c_type} \`e mostrata una schematizzazione delle strutture dati utilizzate all'interno della cache.

%\section{Implementazione}

%Il componente \texttt{Cache\_cmp} pu\`o concettualmente essere diviso in tre parti, ognuna delle quali si interfaccia rispettivamente con DLX, RAM e controllore di memoria.\\
%Per questo motivo si \`e deciso di implementare il componente con 3 process indipendenti, i quali utilizzano segnali interni per sincronizzarsi, pi\`u un quarto processo che si occupa nello specifico di eseguire il rimpiazzamento delle linee.\\

%\subsection{cache\_dlx}

%Il process \texttt{cache\_dlx} si occupa dell'interfacciamento con il DLX eseguendo le operazioni di lettura e scrittura richieste attraverso gli opportuni segnali di controllo .
%I compiti di questo process riguardano quindi i seguenti aspetti:
%\begin{itemize}
%  \item gestione della lettura di dati dalla cache;
%  \item gestione della scrittura dei dati provenienti dal DLX nella cache;
%  \item attivazione del meccanismo di rimpiazzamento di una linea;
%  \item generazione del segnale di ready per il DLX;
%\end{itemize}

%La sensitivity list del processo comprende sia segnali esterni provenienti dal DLX, che segnali interni utilizzati per la sincronizzazione tra i diversi process.\\
%In particolare sono preseti:
%\begin{itemize}
%  \item \texttt{ch\_memrd}: segnale esterno per una richiesta di lettura;
%  \item \texttt{ch\_memwr}: segnale esterno per una richiesta di scrittura;
%  \item \texttt{ch\_reset}: segnale esterno per effettuare il reset del contenuto della cache;
%  \item \texttt{line\_ready}: segnale interno che indica il termine di un rimpiazzamento;
%  \item \texttt{rdwr\_done}: segnale interno che indica, in caso di write-through, il completamento della scrittura in RAM.
%\end{itemize}
% 
%I passi seguito durante una lettura sono:
%\begin{enumerate}
%  \item Lettura dell'indirizzo dal bus separando index, tag e offset;
%  \item Verifica della presenza della linea in cache attraverso \texttt{get\_way()};
%  \item In caso di MISS, attivazione del process per la politica di rimpiazzamento;
%  \item Aggiornamento dei contatori attraverso \texttt{cache\_hit\_on()};
%  \item Lettura del dato dalla cache ed emissione sul bus \texttt{ch\_bdata\_out}.
%\end{enumerate}	

%Per quanto riguarda invece la scrittura, si eseguono le seguenti operazioni:
%\begin{enumerate}
%  \item Lettura dell'indirizzo dal bus separando index, tag e offset; 
%  \item Verifica della presenza della linea in cache attraverso \texttt{get\_way()};
%  \item In caso di MISS, attivazione del process per la politica di rimpiazzamento;
%  \item Scrittura del dato presente in \texttt{ch\_bdata\_in} nella cache;
%  \item Aggiornamento dei contatori attraverso \texttt{cache\_hit\_on()};
%  \item Aggiornamento del bit di stato ed eventuale write-through.
%\end{enumerate}

%
%\lstset{caption=Codice VHDL del process \texttt{cache\_process}, label=DescriptiveLabel}

%\begin{lstlisting}
%Codice del process? Forse diventa un po' lungo...
%\end{lstlisting}

%
%\subsection{cache\_ram}

%Questo process si occupa dell'intefacciamento con la RAM. In particolare, attraverso segnali interni di controllo, possono essere attivati i meccanismi di scrittura e di lettura di un dato.\\

%Durante la realizzazione si \`e ipotizzato che fosse disponibile un segnale di \texttt{ram\_ready} proveniente dall'esterno per indicare il completamento dell'operazione richiesta. Tale segnale \`e importante poich\`e le istruzioni all'interno di uno stesso process vengono eseguite in modo parallelo. Nel nostro caso non sarebbe quindi possibile emettere l'indirizzo per la RAM e leggere immediatamente di seguito i dati sul bus \texttt{ram\_data\_in}.\\

%Nel nostro progetto si \`e supposto che tutti i componenti, compresa la RAM, eseguissero le operazioni in tempo nullo. Tuttavia il segnale \texttt{ram\_ready} diviene indispensabile nel caso in cui si decida di tenere in considerazione i ritardi introdotti da una RAM reale.

%
%\subsection{cache\_snoop}

%Il process \texttt{cache\_snoop} si attiva con il segnale esterno \texttt{ch\_eads} proveniente dal controllore di memoria e consente a quest'ultimo di operare sullo stato delle linee.\\
%In particolare \`e possibile sapere se una determinata linea si trova in cache e se il suo stato \`e MESI\_M.\\
%Tramite il segnale \texttt{ch\_inv} il controllore di memoria pu\`o inoltre forzare l'invalidazione di una particolare linea.\\

%Il process \texttt{cache\_snoop} ha il seguente comportamento: se l'indirizzo richiesto non \`e presente in cache i segnali \texttt{ch\_hit} e \texttt{ch\_hitm} vengono portati al valore logico '0'. In caso contrario il comportamento varia in base allo stato della linea che contiene l'indirizzo:
%\begin{itemize}
%  \item stato MESI\_E: ch\_hit viene portato al valore '1' e la linea passa in stato MESI\_S;
%  \item stato MESI\_S: ch\_hit viene portato al valore '1' e lo stato della linea resta invariato;
%  \item stato MESI\_M: sia ch\_hit che ch\_hitm vengono portati al valore '1', viene forzata la scrittura della linea in RAM e il suo stato vien portato a MESI\_S.
%\end{itemize}

%Nel caso in cui il segnale ch\_inv sia attivo il comportamento resta invariato, ma lo stato della linea diventa sempre MESI\_I. 

%\subsection{cache\_replace}

%I meccasmi per il rimpiazzamento delle linee sono eseguiti dal process \texttt{cache\_replace}. In particolare questo process implementa la politica rimpiazzamento basata sui contatori, stabilendo di volta in volta quale linea rimpiazzare.\\

%Il meccanismo non pu\`o eseguire tutte le operazioni in un unico ciclo, quindi per poter effettuare la sostituzione di una linea in cache con dei dati presenti in RAM \`e stato realizzato un \emph{sequencer} che compie le seguenti operazioni:
%\begin{enumerate}
%  \item determina la riga da sostituire;
%  \item nel caso in cui tale linea sia in stato MESI\_M effettua il write-back sulla RAM;
%  \item attende eventualmente il termine della scrittura;
%  \item attiva il process per la lettura della nuova linea dalla RAM;
%  \item attende il termine della lettura;
%  \item comunica attraverso il segnale interno \texttt{line\_ready} che il rimpiazzamento \`e terminato.
%\end{enumerate}

%\subsection{Comunicazione tra processi}

%I quattro processi si scambiano segnali che consentono la sincronizzazione delle operazioni da svolgere.\\

%\begin{figure}[h!]
%\centering
%\includegraphics[width=\textwidth]{img/cache/collegamenti1.png}
%\caption{Collegamenti tra processi}
%\label{fig:colleg1}
%\end{figure}

%La Fig. \ref{fig:colleg1} mostra come sono collegati i seguenti segnali:
%\begin{itemize}
%  \item \texttt{replace\_line}: attiva il processo che gestisce il rimpiazzamento di una linea;
%  \item \texttt{write\_through}: attiva la scrittura di una linea in stato \texttt{MESI\_S} in memoria RAM;
%  \item \texttt{replace\_write}: attiva la scrittura di una linea da rimpiazzare in stato \texttt{MESI\_M} in memoria RAM;
%  \item \texttt{snoop\_write}: attiva la scrittura di una linea in stato \texttt{MESI\_S} in memoria RAM in seguito ad uno snoop.
%\end{itemize}

%Ogni processo notifica il completamento dell'operazione richiesta attivando un opportuno segnale di ready, come mostrato in Fig. \ref{fig:colleg2}

%\begin{figure}[h!]
%\centering
%\includegraphics[width=\textwidth]{img/cache/collegamenti2.png}
%\caption{Collegamenti tra processi}
%\label{fig:colleg2}
%\end{figure}

%
%\section{Procedure interne}

%Di seguito saranno brevemente descritte le procedure invocate all'interno dei diversi process. \emph{(alcune non ci sono pi\`u e saranno da cavare)}

%
%\subsection{cache\_replace\_line} %(selected\_way: out)}

%Parametri di output:
%\begin{itemize}
%  \item selected\_way: via sulla quale \`e stato caricato il dato rimpiazzato
%\end{itemize}

%Descrizione:
%\begin{enumerate}
%  \item Individua la linea da rimpiazzare, cio\`e quella con \texttt{lru\_counter} massimo
%  \item Controlla se la linea ha stato MESI\_M e in tal caso ne fa il write-back invocando \texttt{ram\_write()}
%  \item Carica il nuovo blocco nella cache sovrascrivendo il vecchio
%  \item Modifica il bit di stato in base al valore di WT\_WB
%  \item Restituisce il numero della via sulla quale \`e presente il dato appena caricato
%\end{enumerate}
%		

%\subsection{cache\_hit\_on} %(hit\_index: in, hit\_way: in)}

%Parametri di input:
%\begin{enumerate}
%  \item \texttt{hit\_index}: indice al quale si \`e verificato l'hit
%  \item \texttt{hit\_way}: via nella quale si \`e verificato l'hit
%\end{enumerate}

%Descrizione:

%Applica la politica di invecchiamento aggiornando i contatori, in particolare:
%\begin{enumerate}
%  \item incrementa i contatori di valore pi\`u basso della via corrente specificata da \texttt{hit\_way}
%  \item resetta il contatore della via corrente
%\end{enumerate}	

%\subsection{cache\_inv\_on} %(inv\_index: in, inv\_way: in)}

%Parametri di input:
%\begin{itemize}
%  \item \texttt{inv\_index}: indice da invalidare
%  \item \texttt{inv\_way}: via da invalidare
%\end{itemize}

%Descrizione:

%Applica la politica di invecchiamento aggiornando i contatori, in particolare:
%\begin{enumerate}
%  \item decrementa i contatori di valore pi\`u alto della via corrente specificata da \texttt{inv\_way}
%  \item porta al valore massimo il contatore della via corrente
%\end{enumerate}	
%	

%\subsection{get\_way} %(index: in, tag: in, way: out) }

%Parametri di input:
%\begin{enumerate}
%  \item \texttt{index}: indice
%  \item \texttt{tag}: tag da controllare
%\end{enumerate}	

%Parametri di output:
%\begin{itemize}
%  \item \texttt{way}: via nella quale \`e presente il dato
%\end{itemize}

%Descrizione:	
%\begin{enumerate}
%  \item Verifica se il dato \`e in cache, cio\`e se esiste una linea con tag uguale a quello specificato il cui stato \`e diverso da \texttt{MESI\_I}
%  \item Se il dato non \`e presente restituisce way = -1
%  \item Se il dato \`e presente restituisce il numero della via
%\end{enumerate}
%	
%	
%%	
%%\subsection{ram\_write} %(tag, index, way)}

%%Parametri di input:
%%\begin{itemize}
%%  \item \texttt{tag}: tag della linea da scrivere
%%  \item \texttt{index}: index della linea da scrivere
%%  \item \texttt{way}: numero di via in cui si trova la linea da scrivere
%%\end{itemize}

%%Descrizione:
%%	1. Costruisce l'indirizzo del blocco a partire da \texttt{tag} e \texttt{index}
%%	2. Scrive i dati contenuti nel blocco sulla RAM

%
%\section{Diagrammi temporali}

%\section{Problematiche principali affrontate}

%(metteri anche tutti i problemi relativi al bus bidirezionale)\\

