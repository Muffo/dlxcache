\clearpage{\pagestyle{empty}\cleardoublepage}
\chapter{Esempio completo di messaggio WSS} 
\label{appendiceWSS} 


L'esempio seguente illustra un caso completo che comprende l'utilizzo di token, firma e cifratura per capire come vengono realizzate le trasformazioni nel messaggio e di come viene modificato l'header nell'utilizzo di Web Service Security.\\
Nel documento il contenuto del body del messaggio � stato cifrato e successivamente il timestamp e l'intero body sono stati firmati digitalmente.\\
L'ordine di queste operazioni si pu� dedurre dall'ordine degli elementi all'interno dell'header \texttt{<wsse:Security>}.
\begin{small}\begin{verbatim}
(001) <?xml version="1.0" encoding="utf-8"?>
(002) <S11:Envelope xmlns:S11="..." xmlns:wsse="..." xmlns:wsu="..."
                            xmlns:xenc="..." xmlns:ds="...">
(003)   <S11:Header>
(004)      <wsse:Security>
(005)         <wsu:Timestamp wsu:Id="T0">
(006)           <wsu:Created>
(007)                   2001-09-13T08:42:00Z</wsu:Created>
(008)         </wsu:Timestamp>
(009)
(010)         <wsse:BinarySecurityToken
                     ValueType="...#X509v3"
                     wsu:Id="X509Token"
                     EncodingType="...#Base64Binary">
(011)         MIIEZzCCA9CgAwIBAgIQEmtJZc0rqrKh5i...
(012)         </wsse:BinarySecurityToken>
(013)         <xenc:EncryptedKey>
(014)             <xenc:EncryptionMethod Algorithm=
                        "http://www.w3.org/2001/04/xmlenc#rsa-1_5"/>
(015)             <ds:KeyInfo>
(016)                <wsse:KeyIdentifier
                         EncodingType="...#Base64Binary"
                   ValueType="...#X509v3">MIGfMa0GCSq...
(017)                </wsse:KeyIdentifier>
(018)             </ds:KeyInfo>
(019)             <xenc:CipherData>
(020)                <xenc:CipherValue>d2FpbmdvbGRfE0lm4byV0...
(021)                </xenc:CipherValue>
(022)             </xenc:CipherData>
(023)             <xenc:ReferenceList>
(024)                 <xenc:DataReference URI="#enc1"/>
(025)             </xenc:ReferenceList>
(026)         </xenc:EncryptedKey>
(027)         <ds:Signature>
(028)            <ds:SignedInfo>
(029)               <ds:CanonicalizationMethod
            Algorithm="http://www.w3.org/2001/10/xml-exc-c14n#"/>
(030)               <ds:SignatureMethod
            Algorithm="http://www.w3.org/2000/09/xmldsig#rsa-sha1"/>
(031)               <ds:Reference URI="#T0">
(032)                  <ds:Transforms>
(033)                     <ds:Transform
            Algorithm="http://www.w3.org/2001/10/xml-exc-c14n#"/>
(034)                  </ds:Transforms>
(035)                  <ds:DigestMethod
            Algorithm="http://www.w3.org/2000/09/xmldsig#sha1"/>
(036)                  <ds:DigestValue>LyLsF094hPi4wPU...
(037)                   </ds:DigestValue>
(038)               </ds:Reference>
(039)               <ds:Reference URI="#body">
(040)                  <ds:Transforms>
(041)                     <ds:Transform
            Algorithm="http://www.w3.org/2001/10/xml-exc-c14n#"/>
(042)                  </ds:Transforms>
(043)                  <ds:DigestMethod
            Algorithm="http://www.w3.org/2000/09/xmldsig#sha1"/>
(044)                  <ds:DigestValue>LyLsF094hPi4wPU...
(045)                   </ds:DigestValue>
(046)               </ds:Reference>
(047)            </ds:SignedInfo>
(048)            <ds:SignatureValue>
(049)                     Hp1ZkmFZ/2kQLXDJbchm5gK...
(050)            </ds:SignatureValue>
(051)            <ds:KeyInfo>
(052)                <wsse:SecurityTokenReference>
(053)                    <wsse:Reference URI="#X509Token"/>
(054)                </wsse:SecurityTokenReference>
(055)            </ds:KeyInfo>
(056)         </ds:Signature>
(057)      </wsse:Security>
(058)   </S11:Header>
(059)   <S11:Body wsu:Id="body">
(060)      <xenc:EncryptedData
                  Type="http://www.w3.org/2001/04/xmlenc#Element"
                  wsu:Id="enc1">
(061)         <xenc:EncryptionMethod
        Algorithm="http://www.w3.org/2001/04/xmlenc#tripledes-cbc"/>
(062)         <xenc:CipherData>
(063)            <xenc:CipherValue>d2FpbmdvbGRfE0lm4byV0...
(064)            </xenc:CipherValue>
(065)         </xenc:CipherData>
(066)      </xenc:EncryptedData>
(067)   </S11:Body>
(068) </S11:Envelope>
\end{verbatim}\end{small}
Concentriamoci prima sulle linee (003-058) che contengono gli header del messaggio SOAP, in particolare le linee (004-057) riguardano tutto un unico elemento \texttt{<wsse:Security>} per il destinatario finale.\\ 
Le linee (005-008) contengono le informazioni del timestamp che in questo caso contiene la data di creazione dell'header corrente.\\
Le linee (010-012) contengono un security token associate al messago. In questo caso si tratta di un certificato X.509 (che vedremo nei dettagli in seguito). La linea (011) ne � la rappresentazione con il tradizionale encoding Base64.\\

Le linee (013-026) specificano la chiave usata per cifrare il body del messaggio. Siccome si tratta di una chiame simmetrica, questa viene trasmessa a sua volta in forma cifrata.\\
La linea (014) definisce l'algoritmo utilizzato per cifrare la chiave.\\
Le linee (015-018) specificano l'identificativo della chiave usata per cifrare la chiave simmetrica.\\
Le linee (019-022) contengono infine la rappresentazione cifrata della chiave simmetrica.
Le linee (023-025) contengono il riferimento alla parte di messaggio che � stata cifrata con questa chiave simmetrica. In questo caso si tratta solo del body (\texttt{Id="enc1"}).

Le linee (027-056) contengono la firma digitale. La firma � basata sul certificato X.509.\\
Le linee (028-047) indicano cosa � stato firmato e con che metodo, in particolare la linea (031) contiene un riferimento al timestamp e la linea (039) contiene un riferimento al body.\\
Le linee (048-050) contengono la rappresentazione della firma vera e propria.\\
Le linee (052-054) indicano la chiave usata per la firma, ossia il certificato X.509 incluso nel messaggio, in particolare nella linea (053) compare il riferimento al token contenuto nelle righe  (010-012).

Il body del messaggio � contenuto nelle linee (059-067).\\
Le linee (060-066) representano i metadati relativi al messaggio cifrato usati da XML Encryption.\\
In particolare la linea (060) contiene un riferimento \texttt{wsu:Id="enc1"} che  viene puntato dalla chiave (dalla linea (024)) e indica che tutto l'elemento � stato sostituito con questo blocco di informazioni usate da XML Encryption.\\
La linea (061) specifica l'algoritmo di cifratura (Triple-DES).\\
Le linee (063-064) contengono il messaggio vero e proprio, risultato della cifratura.\\
